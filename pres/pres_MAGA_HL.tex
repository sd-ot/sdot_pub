\documentclass[aspectratio=169]{beamer}
\setbeamertemplate{navigation symbols}{} % don't use navigation tools on slides
% \usetheme{LMT}

\usepackage[utf8]{inputenc}
\usepackage{pdfpc-commands}
\usepackage{multimedia}
\usepackage{listings}
\usepackage{default}

\setbeamersize{text margin left=0.6cm,text margin right=0.6cm}
\setbeamercolor{frametitle}{fg=black}
\setbeamercolor{section in toc}{fg=black}
\setbeamertemplate{frametitle}{\color{black}\bfseries\insertframetitle\par\vskip-6pt{\color{gray}\hrulefill}}

\lstset{language=C++,
  basicstyle=\ttfamily,
  keywordstyle=\color{blue}\ttfamily,
  stringstyle=\color{red}\ttfamily,
  commentstyle=\color{green}\ttfamily,
  morecomment=[l][\color{magenta}]{\#}
}

\AtBeginSection[]{
  \begin{frame}{Sommaire}
    \tableofcontents[currentsection]
  \end{frame} 
}

\begin{document}

\begin{frame}
    \begin{center}
        {\huge High Performance Computing of Power Diagrams}

        \bigskip
        {\large Applications to Semi-Discrete Optimal Transport}
      
        \vfill
        {(MAGA days, November 21, 2019, Hugo Leclerc)}
    \end{center}
\end{frame}

% ---------------------------------------------------------------------------------------
\section{}

\begin{frame}
    \frametitle{The world needs power diagrams !}

    \begin{minipage}{0.5\textwidth}
        Problem to solve: what is the optimal way to transport a density to a set of diracs with equal mass ? 
        
        \bigskip
        Quadratic cost (euclidian distance) $\Rightarrow$ \textit{attributions} are defined by power diagrams.
    \end{minipage}
    \kern 0.5cm
    \begin{minipage}{0.45\textwidth}
        \begin{center}
        \end{center}
    \end{minipage}

    
\end{frame}

\begin{frame}
    \frametitle{$\Rightarrow$ The world needs libraries to compute power diagrams !}

    Libraries for power diagrams are not common, and most of them are focused on obtaining the exact connectivity.
    
    \vfill
    Could be parallelized but not in a natural way.
    
    \vfill
    Exact connectivity is not a necessity for optimal transportation (one compute integrals where connectivity issues vanish).
\end{frame}

% \begin{frame}
%     \frametitle{Parcours en bref}
%  
%     \begin{minipage}{0.5\textwidth}
%         \begin{itemize}
%             \item Ingénieur de recherche, LMT/CMAP \\ 
%                 {\color{gray}
%                     Stratégies de calculs / traitement de données expérimentales... \\[0.15cm]
%                     Applications~: identification de lois de comportement, réduction de dimensionnalité, décomposition de domaine, contrôle d'erreur, corrélation, reconstruction, etc...
%                 }
%                 
%             \item Thèse en mécanique \\ 
%                 {\color{gray} Modél., simu. et expé. frittage par laser}
%                 
%             \item Demomaker \\
%                 {\color{gray} Assembleur, connaissance des machines}
%       \end{itemize}
%     \end{minipage}
%     \kern 0.5cm
%     \begin{minipage}{0.45\textwidth}
%         \begin{center}
%         \only<1>{
%             \includegraphics[height=0.22\textwidth]{img/serpent_cache.png} \kern 0.1cm
%             \includegraphics[height=0.22\textwidth]{img/bz_2d.png}
% 
%             \kern 0.2cm
%             \includegraphics[width=0.4\textwidth]{img/few_projections.png} \kern 0.1cm
%             \includegraphics[width=1.0\textwidth]{img/tomo_du.png} % \kern 0.1cm
%             
%             \kern 0.1cm
%             \includegraphics[width=1.0\textwidth]{img/binary_reco.png}
%         }
%         \only<2>{
%             \includegraphics[width=0.35\textwidth]{img/frittage_photo_four.png} \kern 0.1cm
%             \includegraphics[width=0.55\textwidth]{img/frittage_principe.png}
%             
%             \kern 0.5cm
%             \includegraphics[width=0.98\textwidth]{img/frittage_sep.png}
%             
%             \kern 0.25cm
%             \includegraphics[width=0.75\textwidth]{img/frittage_simu_une_particle.png} \kern 0.1cm
%             \includegraphics[width=0.2\textwidth]{img/frittage_simu_3d.png}
%         }
%         \only<3>{
%             \includegraphics[width=\textwidth]{img/atari_st_forever.png} \\
%             (1991, Checkpoint)
%         }
%         \end{center}
%     \end{minipage}
% \end{frame}

 
% ---------------------------------------------------------------------------------------
\section{Conclusions et perspectives}

\begin{frame}
    \frametitle{Conclusions}

    Il est possible de rapprocher \textit{encore} la programmation des maths appliquées
    \begin{itemize}
        \item pour perdre moins de temps
        \item pour mieux séparer les difficultés
        \item et moins se déformer l'esprit
    \end{itemize}
    
    \vfill
    Illustration avec l'évaluation paresseuse~: 
    \begin{itemize}
        \item simplification extrême des écritures 
        \item avec sémantique significativement plus précise
        \item adaptations automatiques et évolutives en fonction du hardware
        \item programmation symbolique
    \end{itemize}
\end{frame}


\begin{frame}
    \frametitle{Perspectives}

    \begin{itemize}
        \item Différentiations (positions initiales, bords, ...) et adaptations pour le transport optimal (semi-discret)
        \vfill
        \item Simplification des représentations pour les reconstructions tomographiques
        \vfill
        \item Plus grande palette de schémas d'intégration (collant aux équations)
        \vfill
        \item Extraction de propriétés mathématiques
    \end{itemize}
\end{frame}


\end{document}
